
\documentclass[twoside]{article}
\setlength{\oddsidemargin}{0.05 in}
\setlength{\evensidemargin}{-0.05 in}
\setlength{\topmargin}{-0.6 in}
\setlength{\textwidth}{6.5 in}
\setlength{\textheight}{9.5 in}
\setlength{\headsep}{0.25 in}
\setlength{\parindent}{0 in}
\setlength{\parskip}{0.1 in}

\usepackage{amsmath,amsfonts,graphicx}

\newcounter{lecnum}
\renewcommand{\thepage}{\thelecnum-\arabic{page}}
\renewcommand{\thesection}{\thelecnum.\arabic{section}}
\renewcommand{\theequation}{\thelecnum.\arabic{equation}}
\renewcommand{\thefigure}{\thelecnum.\arabic{figure}}
\renewcommand{\thetable}{\thelecnum.\arabic{table}}
%
% The following macro is used to generate the header.
%
\newcommand{\lecture}[2]{
   \pagestyle{myheadings}
   \thispagestyle{plain}
   \newpage
   \setcounter{lecnum}{#1}
   \setcounter{page}{1}
   \noindent
   \begin{center}
   \mbox{
       \vbox{\vspace{2mm}
         \hbox {\leftline{\Large LECTURE #1: \hfill}}
         \vspace{3mm}
         \hbox {\leftline{\Large #2 \hfill}}
         \vspace{4mm}
         \hrule
         \vspace{3mm}
         \hbox to 6.4in { {H. Xiao \hfill NOVEMBER 2012} }
         \vspace{3mm}}
        }
   \end{center}
   \markboth{LECTURE #1: #2}{LECTURE #1: #2}
   \vspace*{4mm}
}

\renewcommand{\cite}[1]{[#1]}
\def\beginrefs{\begin{list}%
        {[\arabic{equation}]}{\usecounter{equation}
         \setlength{\leftmargin}{2.0truecm}\setlength{\labelsep}{0.4truecm}%
         \setlength{\labelwidth}{1.6truecm}}}
\def\endrefs{\end{list}}
\def\bibentry#1{\item[\hbox{[#1]}]}

%Use this command for a figure; it puts a figure in wherever you want it.
%usage: \fig{NUMBER}{SPACE-IN-INCHES}{CAPTION}

\newcommand{\fig}[3]{
    		\vspace{#2}
			\begin{center}
			Figure \thelecnum.#1:~#3
			\end{center}
	}
% Use these for theorems, lemmas, proofs, etc.

\def\Z{{\mathbb Z}}
\def\N{{\mathbb N}}
\def\R{{\mathbb R}}
\def\C{{\mathbb C}}
\def\F{{\mathbb F}}
\def\L{{\mathcal L}}
\def\P{{\mathcal P}}
\def\tr{\mathop{\rm tr}}
\def\im{\mbox{im}}

\newcommand{\V}{{\bf V}}
\newcommand{\W}{{\bf W}}
\newcommand{\U}{{\bf U}}
\newcommand{\ip}[2]{\langle {#1}, {#2} \rangle}

\newtheorem{theorem}{Theorem}[lecnum]
\newtheorem{lemma}[theorem]{Lemma}
\newtheorem{proposition}[theorem]{Proposition}
\newtheorem{claim}[theorem]{Claim}
\newtheorem{corollary}[theorem]{Corollary}
\newtheorem{definition}[theorem]{Definition}

\newenvironment{proof}{{\it Proof.}}{\hfill\rule{2mm}{2mm}}


\begin{document}

\lecture{1}{Lattice Theory}

\section{Lattice}
Geometrically, a lattice can be defined as the set of intersection point of an infinite, regular, but not necessarily orthogonal n-dimensional grid. For example, the set of integer vectors $\Z^n$ is a lattice. In theoretical computer science, lattices are usually represented by a generating basis.

\begin{definition}[Lattice]
Given $m$ linearly independent vectors $b_1,b_2,\dots,b_m\in \R^n$, the lattice generated by them is defined as $L(b_1,b_2,\dots,b_m)=\{\sum x_ib_i | x_i\in \Z\}$. We refer to $b_1,b_2,\dots,b_m$ as a basis of the lattice. We say that the rank of the lattice is $m$ and its dimension is $n$. If $m=n$, the lattice is called a full-rank lattice. A lattice is usually denoted by $\Lambda$.
\end{definition}

Alternative Definition of Lattices
\begin{definition}[Lattice]
A lattice is a discrete additive subgroup of $\R^n$ generated by all the integer combinations of some basis.
\end{definition}

Notice the similarity between the definition of a lattice and the definition of vector space generated by $b_1,b_2,\dots,b_n$.
$$\mbox{span}\{b_1,b_2,\dots,b_m\}=\{\sum_{i=1}^{m} x_ib_i | x_i\in \R\}$$
One difference is that in a vector space you can combine the basis vectors with arbitrary real coefficients, while in a lattice only integer coefficients are allowed, resulting in a discrete set of points.

Another difference between lattices and vector spaces is that vector spaces always admit an orthogonal basis. This is not true for lattices.

Given a  lattice $\Lambda$, there are infinitely many different choices of lattice basis. Let $B$ be a nonsingular matrix with one basis of $\Lambda$ as the columns of it. One can obtain another basis $C$ by a unimodular transformation(multiplying the basis vectors by a square matrix with integer entries and determinant plus or minus one.), then $|\det B|=|\det C|$. So this number is independent of the choice of the basis, and is called the \emph{determinant} of $\Lambda$, denoted by $\det\Lambda$. It is equal to the volume of the parallelepiped $\{x_1b_1+\dots+x_nb_n|0\leq x_i<1 \mbox{~for~} i=1,\dots,n\}$

\textbf{Vector space}

\begin{definition}[Fundamental Parallelepiped]
Given $m$ linearly independent vectors $b_1,b_2,\dots,b_m\in \R^n$, their fundamental parallelepiped is defined as
$$\{\sum_{i=1}^{m}x_ib_i|x_i\in\R, 0\leq x_i<1\}$$
\end{definition}

vol$=|\det{B}|$

\begin{theorem}[Hadamard inequality]
$\det\Lambda\leq \lVert b_1\rVert \lVert b_2\rVert \cdots \lVert b_m \rVert,$
where $\lVert \cdot\rVert$ denotes Euclidean norm ($\lVert x\rVert=\sqrt{x^Tx}$).
\end{theorem}

\section{Gram-Schmidt }

Projection

\section{LLL lattice basis reduction algorithm}

Motivation

\section{Applications}


\section*{References}
\beginrefs
\bibentry{CW87}{\sc D.~Coppersmith} and {\sc S.~Winograd},
``Matrix multiplication via arithmetic progressions,''
{\it Proceedings of the 19th ACM Symposium on Theory of Computing},
1987, pp.~1--6.
\endrefs

% **** THIS ENDS THE EXAMPLES. DON'T DELETE THE FOLLOWING LINE:

\end{document}






